\documentclass[]{article}
\usepackage{enumerate}
\usepackage{graphicx}
\usepackage{amsmath}
\usepackage{amssymb}
\usepackage{todonotes}
\usepackage{rotating}
\usepackage{tabularx}
% \usepackage{algorithm2e} % [ruled,linesnumbered]
%opening
\title{SA Exercise 3}
\author{Kai Biermeier (7028629), Hassen Smaoui (6782765), \\ Daniel Wecker (7005021)}

\begin{document}
	
	\maketitle
	\section*{Exercise 1}
	a)\\
	Equation System:\\
	(i,r,x)\\
	$CP_1$=$\iota$=($\top$, $\top$, $\top$)\\
	$CP_2$=$\phi$($CP_1$)=($\top$, $\top$, 0 )\\
	$CP_3$=$\phi$($CP_2$)=($\top$, 0, 0 )\\
	$CP_4$=$\phi$($CP_3$)=(6, 0, 0 )\\
	$CP_5$=$\phi$($CP_4$)=(6, $\top$, 0 )\\
	$CP_6$=$\phi$($CP_4$)=(6, $\top$, 0 )\\
	$CP_7$=$\phi$($CP_6$) $\sqcup$ $\phi$($CP_5$)= (6, $\top$, 0 )\\
	b)\\
	$mop(l_7)$=$\phi_{[1,2,3,4,5,7]}$$\sqcup$$\phi_{[1,2,3,4,6,7]}$=(6, $\top$, 0 )$\sqcup$(6, $\top$, 0 )= (6, $\top$, 0 )\\
	c)\\
	IDEAL($l_7$) is equal to mop($l_7$) in this case, because there exists not case (path) where i or j are reassigned since 6 and 0 are assigned to them. Moreover r will in any case be reeassigned with 3 (r=0<5). Therefore r is not constant.
	\newpage
	
	\section*{Exercise 2}
	a) 
	Prove that $(D^n,\sqsubseteq^n)$ is a complete latice:
	Since $(D,\sqsubseteq)$ is a latice it is also a partial order. Therefore two tuples a,b $\in$ $D^n$ are also partitially ordered by pointwise applying the $\sqsubseteq$. Since we define $\sqsubseteq^n$ as the pointwise computation of $\sqsubseteq$ for a and b $(D^n,\sqsubseteq^n)$ is a partitial order. As we can execute the $\sqcup$ and $\sqcap$ also pointwise for a and b (because $(D,\sqsubseteq)$ is a latice), we can compute in that way also the glb and lup for $(D^n,\sqsubseteq^n)$.\\
	Therefore $(D^n,\sqsubseteq^n)$ is a partial order with glb and lup\\
	$\Rightarrow (D^n,\sqsubseteq^n)$ is a complete latice
	b)\\
	\begin{sidewaystable}
		\begin{tabularx}{\textwidth}{|l|X|X|X|X|X|X|X|X|}
			\hline 
			& $AI_0$  & $AI_1$ & $AI_2$ & $AI_3$ & $AI_4$ &  $AI_5$ & $AI_6$ & Worklist \\ 
			\hline 
			Init & \{(x,?),(i,?),\newline(r,?)\} & $\bot_D$ & $\bot_D$ & $\bot_D$ & $\bot_D$ & $\bot_D$ & $\bot_D$ & (0,1),(1,2),\newline(0,2),(2,3),\newline(3,4),(4,5),\newline(5,6),(6,4)\\ 
			\hline 
			1 & \{(x,?),(i,?),\newline(r,?)\} & \{(x,?),(i,?),\newline(r,?)\} & $\bot_D$ & $\bot_D$ & $\bot_D$ & $\bot_D$ & $\bot_D$ & (1,2),(0,2),\newline(2,3),(3,4),\newline(4,5),(5,6),\newline(6,4)\\ 
			\hline  
			2 & \{(x,?),(i,?),\newline(r,?)\} & \{(x,?),(i,?),\newline(r,?)\} & \{(x,1),\newline(i,?),(r,?)\} & $\bot_D$ & $\bot_D$ & $\bot_D$ & $\bot_D$ &  (0,2),(2,3),\newline(3,4),(4,5),\newline(5,6),(6,4)\\ 
			\hline 
			3 & \{(x,?),(i,?),\newline(r,?)\} & \{(x,?),(i,?),\newline(r,?)\} & \{(x,1),\newline(x,?),\newline(i,?),(r,?)\} & $\bot_D$ & $\bot_D$ & $\bot_D$ & $\bot_D$ & (2,3),(3,4),\newline(4,5),(5,6),\newline(6,4)\\ 
			\hline 
			4 & \{(x,?),(i,?),\newline(r,?)\} & \{(x,?),(i,?),\newline(r,?)\} & \{(x,1),(x,?),\newline(i,?),\newline(r,?)\} & \{(x,1),(x,?),\newline\newline(i,2),(r,?)\} & $\bot_D$ & $\bot_D$ & $\bot_D$ & 
			(3,4),(4,5),\newline(5,6),(6,4)\\ 
			\hline 
			5 & \{(x,?),(i,?),\newline(r,?)\} & \{(x,?),(i,?),\newline(r,?)\} & \{(x,1),(x,?),\newline(i,?),\newline(r,?)\} & \{(x,1),(x,?),\newline(i,2),\newline(r,?)\} & \{(x,1),(x,?)\newline,\newline(i,2),(r,3)\} & $\bot_D$ & $\bot_D$ & (4,5),(5,6),\newline(6,4)\\ 
			\hline
			6 & \{(x,?),(i,?),\newline(r,?)\} & \{(x,?),(i,?),\newline(r,?)\} & \{(x,1),(x,?),\newline(i,?),(r,?)\} & \{(x,1),(x,?),\newline(i,2),\newline(r,?)\} & \{(x,1),(x,?),\newline(i,2),(r,3)\} & \{(x,1),(x,?),\newline(i,2),\newline(r,3)\} & $\bot_D$ & 
			(5,6),(6,4)\\ 
			\hline
			7 & \{(x,?),(i,?),\newline(r,?)\} & \{(x,?),(i,?),\newline(r,?)\} & \{(x,1),(x,?),\newline(i,?),(r,?)\} & \{(x,1),(x,?),\newline(i,2),\newline(r,?)\} & \{(x,1),(x,?),\newline(i,2),(r,3)\} & \{(x,1),(x,?),\newline(i,2),(r,3)\} & \{(x,1),(x,?),\newline(i,2),(r,5),\newline(r,5)\} & (6,4)\\ 
			\hline
			8 & \{(x,?),(i,?),\newline(r,?)\} & \{(x,?),(i,?),\newline(r,?)\} & \{(x,1),(x,?),\newline(i,?),(r,?)\} & \{(x,1),(x,?),\newline(i,2),\newline(r,?)\} & \{(x,1),(x,?),\newline(i,6),(r,3)\} & \{(x,1),(x,?),\newline(i,2),(r,3)\} & \{(x,6),(x,?),\newline(i,2),(r,5),\newline(r,5)\} & (4,5),(5,6),\newline(6,4)\\ 
			\hline
			9 & \{(x,?),(i,?),\newline(r,?)\} & \{(x,?),(i,?),\newline(r,?)\} & \{(x,1),(x,?),\newline\newline(i,?),(r,?)\} & \{(x,1),(x,?),\newline(i,2),\newline(r,?)\} & \{(x,1),(x,?),\newline\newline(i,6),(r,3)\} & \{(x,1),(x,?)\newline,\newline(i,6),(r,3)\} & \{(x,6),\newline(i,2),(r,5),\newline(r,5)\} & (5,6),(6,4)\\ 
			\hline
			10 & \{(x,?),(i,?),\newline(r,?)\} & \{(x,?),(i,?),\newline(r,?)\} & \{(x,1),(x,?),\newline\newline(i,?),(r,?)\} & \{(x,1),(x,?)\newline,\newline(i,2),\newline(r,?)\} & \{(x,1),(x,?),\newline\newline(i,6),(r,3)\} & \{(x,1),(x,?)\newline,\newline(i,6),(r,3)\} & \{(x,6),\newline(i,6),(r,5),\newline(r,5)\} & (6,4)\\ 
			\hline
			11 & \{(x,?),(i,?),\newline(r,?)\} & \{(x,?),(i,?),\newline(r,?)\} & \{(x,1),(x,?),\newline\newline(i,?),(r,?)\} & \{(x,1),(x,?)\newline,\newline(i,2),\newline(r,?)\} & \{(x,1),(x,?),\newline\newline(i,6),(r,3)\} & \{(x,1),(x,?)\newline,\newline(i,6),(r,3)\} & \{(x,6),\newline(i,6),(r,5),\newline(r,5)\} & \\ 
			\hline
		\end{tabularx}
	\end{sidewaystable}
\end{document}